\section{Context}

The utilization of hydrogen stands as a pivotal step towards combating climate change through the decarbonization of industrial processes, transportation and the seamless integration of variable renewable energy sources into the electricity grid \citep{shukla2022climate}.
Presently, there is a significant global upsurge in the demand for hydrogen, highlighting its increasing importance across various sectors \citep{iea_global_2022}. Especially for industry, hydrogen can be converted into ammonia in order to be primarily used for fertilizers. 
\\
With this rise in demand, low-emission hydrogen production is expected to grow from 1 to about 20 Mtons/year \citep{iea_global_2022} leading to a constant expansion of the pipeline of projects aiming at fostering low-emission hydrogen production. In order to reduce the specific production costs, gradually larger projects are announced with capacities already well exceeding 300MW planned in Germany, China and Australia \citep{iea_hydrogen_2023}.
\\
Concurrently, standards regarding the definiton of low-emission hydrogen are being discussed, initiating the need to anticipate new production constraints for the project holders. For instance, the \cite{green_hydrogen_organisation_green_2023} proposes to define a standard requiring that green hydrogen projects operate at less than 1 ton of CO2 emitted per ton of H2 produced taken as an average over a 12-months period.
\\

In pursuit of enhancing the competitiveness and integration of low-emission hydrogen within the broader energy landscape, extensive research has been undertaken. This integration marks a paradigm shift in scientific understanding with the emergence of smart energy systems \citep{lund_smart_2017}, which employ multiple energy vectors.
Hydrogen can be integrated into such systems mainly through electrolysis processes to boost power-to-gas production  while the waste heat from these processes can be valorized through district heating for e.g. \citep{guelpa_towards_2019}.
The literature presents studies of numerous multi-vector systems. \cite{gabrielli_optimal_2018} delved into the optimal design of energy systems encompassing both electricity and heat demands, thereby facilitating the integration of a substantial proportion of renewable electricity through hydrogen production and storage mechanisms.
\cite{haratyk_nuclear-renewables_2012} conducted an in-depth exploration of a nuclear-renewable energy system focusing on both hydrogen and electricity production, adressing particularly the utilization of electrolyzer partial and overload modes and hydrogen storage mechanisms to effectively balance the inherent variability of wind energy.
\\
Moreover, as planned for the announced projects \citep{iea_hydrogen_2023}, low-emission hydrogen production is generally associated to storage capacities and particularly large ones. The interest for large storage is also denoted in the literature in the field of power-to-gas-to-power. \cite{gabrielli_seasonal_2020} shed light on the immense potential of extensive hydrogen storage solutions, such as salt caverns, in mitigating CO2 emissions from energy grids characterized by high variable renewable energy production shares.
\cite{clerjon_matching_2019} uses an innovative signal decomposition approach \citep{CLERJON2022122799} to show that seasonal electricity storage is particularly difficult to ensure and the only economically viable solution appears to be H2 storage in salt caverns.\\

As stated beforehand, the future industrial green H2 production plant will be large and probably associated to large salt cavern. The latter means that an optimized operational management, i.e. an Energy Management Systems (EMS), should allow for consequent financial savings. However, this EMS will face the following challenges: 
\begin{itemize}
    \item The horizon of prediction of renewable productions does not exceed few days while the system is equipped with a storage which capacity exceeds weeks of production. This time-scale mismatch leads to the necessity of adding long-term management capabilities to the EMS.
    \item Within the industrial context, the green H2 must be produced respecting constraints such as CO2 content \citep{green_hydrogen_organisation_green_2023} or rate of production to match the low flexibility of downstream processes such as Haber-Bosch ammonia production reactor \citep{shamiri_modeling_2021}.
\end{itemize}
In the present work, an EMS that accounts for long-term management and industrial constraints is developed in order to evaluate the financial savings obtained on a large-scale plant.
