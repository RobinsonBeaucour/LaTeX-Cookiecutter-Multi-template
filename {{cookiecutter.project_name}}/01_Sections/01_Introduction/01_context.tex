\subsection{Context}

The utilization of hydrogen stands as a pivotal step towards combating climate change through the decarbonization of industrial processes, transportation and the seamless integration of variable renewable energy sources into the electricity grid \citep{shukla2022climate}.
Presently, there is a significant global upsurge in the demand for hydrogen, highlighting its increasing importance across various sectors \citep{iea_global_2022}. Especially for industry, hydrogen can be converted into ammonia in order to be primarily used for fertilizers. 
\\

As stated beforehand, the future industrial green H2 production plant will be large and probably associated to large salt cavern. The latter means that an optimized operational management, i.e. an Energy Management Systems (EMS), should allow for consequent financial savings. However, this EMS will face the following challenges: 
\begin{itemize}
    \item The horizon of prediction of renewable productions does not exceed few days while the system is equipped with a storage which capacity exceeds weeks of production. This time-scale mismatch leads to the necessity of adding long-term management capabilities to the EMS.
    \item Within the industrial context, the green H2 must be produced respecting constraints such as CO2 content \citep{green_hydrogen_organisation_green_2023} or rate of production to match the low flexibility of downstream processes such as Haber-Bosch ammonia production reactor \citep{shamiri_modeling_2021}.
\end{itemize}
In the present work, an EMS that accounts for long-term management and industrial constraints is developed in order to evaluate the financial savings obtained on a large-scale plant.
