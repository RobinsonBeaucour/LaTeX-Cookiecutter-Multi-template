\subsection{State-of-the-art}
The existing literature offers a rich array of methodologies for constructing EMS \citep{weitzel_energy_2018}. As summarized by \cite{alabi_strategic_2023}, the two advanced EMS methodologies that can cope with the dynamic variations of smart energy systems are Model Predictive Control (MPC) and Reinforcement Learning (RL). The latter approaches are illustrated by \cite{stoffel_evaluation_2023}, in the context of building energy management systems, who compared traditional reactive expert rules to model-free RL, MPC based on white-box, gray-box and black-box modeling and approximate MPC. While RL seems promising, there are still scientific challenges to overcome such as its lake of explicability, the effective management of specific constraints and the necessary consequent offline training on precise simulators. Those are probably the reasons why the studies that have probed the domain of EMS tailored specifically to hydrogen production in industrial settings are based on MPC.
\\
Noteworthy among these, \cite{klyapovskiy_optimal_2021} showcased the considerable potential of EMS in increasing the economic performance of industrial facilities reliant on hydrogen. Their approach involved the utilization of hydrogen and electricity, partially sourced from local renewables, emphasizing the pivotal role of EMS in optimizing such intricate energy systems, all executed within the framework of a Mixed-Integer Linear Programming (MILP) model. Expanding on this, their approach assesses energy management across four representative days, providing a planning horizon with a renewable production forecast of 24 hours. Such an approach however lacks the management of long term phenomena.
\\
Regarding EMS strategies including long term management but not in the field of hydrogen production, \cite{cuisinier_new_2022,cuisinier_impact_2023} presented a rolling horizon approach based on a MILP model that combines the short term, evaluated hourly, and the long term, evaluated at an arbitrarily longer interval.
The sole decision variable for the long term becomes the variation in the State of Energy (SOE) of the long-term storage, a thermal pit storage associated to district heating in their case. This variable is employed by Cost Functions (CF), which represent the operational cost of the system for a given time interval and for a variation of storage SOE. These CF are precalculated based on numerous MILP models over Representative Periods (RP) of the long term horizon.
\\

Based on the analysis of the state-of-the-art, none of the papers examined cope with industrial constraints such as maximal CO2 content of the produced H2 or minimal H2 rate of production. The most recent and advanced approaches for EMS tailored specifically to hydrogen production in industrial settings are based on hierarchical approaches, that may have difficulties to cope with such constraints. In the field of district heating, the methodological continuous approach of \cite{cuisinier_new_2022} is promising in order to both cope with the long term management of the salt cavern and these industrial constraints.
