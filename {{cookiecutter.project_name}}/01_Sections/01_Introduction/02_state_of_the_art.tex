\subsection{State-of-the-art}
The existing literature offers a rich array of methodologies for constructing EMS \citep{weitzel_energy_2018}. As summarized by \cite{alabi_strategic_2023}, the two advanced EMS methodologies that can cope with the dynamic variations of smart energy systems are Model Predictive Control (MPC) and Reinforcement Learning (RL). The latter approaches are illustrated by \cite{stoffel_evaluation_2023}, in the context of building energy management systems, who compared traditional reactive expert rules to model-free RL, MPC based on white-box, gray-box and black-box modeling and approximate MPC. While RL seems promising, there are still scientific challenges to overcome such as its lake of explicability, the effective management of specific constraints and the necessary consequent offline training on precise simulators. Those are probably the reasons why the studies that have probed the domain of EMS tailored specifically to hydrogen production in industrial settings are based on MPC.
\\
Noteworthy among these, \cite{klyapovskiy_optimal_2021} showcased the considerable potential of EMS in increasing the economic performance of industrial facilities reliant on hydrogen. Their approach involved the utilization of hydrogen and electricity, partially sourced from local renewables, emphasizing the pivotal role of EMS in optimizing such intricate energy systems, all executed within the framework of a Mixed-Integer Linear Programming (MILP) model. Expanding on this, their approach assesses energy management across four representative days, providing a planning horizon with a renewable production forecast of 24 hours. Such an approach however lacks the management of long term phenomena.
\\
When adressing specifically long-term management with MPC, the methodological approches can be classified into 2 categories: i) the hierarchical approach and ii) the continuous approach. For the hierarchical approach, the EMS uses 2 different models solved on different time horizons (short/long) and meshes (fine/coarse). It is called hierarchical because the long-term model outputs are used to set constraints and/or objectives to the short-term model. The continuous approach is however based on a single model with different strategies to tackle the long-term phenomena.\\

Among hierarchical approaches, \cite{petrollese_real-time_2016} and \cite{weber_model_2022} have explored EMS based on a combination of Optimal Generation Scheduling (OGS) and MPC. The OGS provides reference trajectories for energy storage levels, which the MPC aims to follow with incentives in the objective function. The reference trajectory is updated recursively when the energy system state changes.
\\
In \cite{petrollese_real-time_2016}, the system is a renewable hydrogen-based microgrid. It includes various energy storage systems, and the OGS provides objectives for battery and hydrogen storage levels, taking into account uncertainties in weather and load forecasting. The MPC component handles the real-time management, adjusting actions to meet these objectives while responding to current conditions.
\\
In \cite{weber_model_2022}, the EMS focuses on building climate control with seasonal energy storage, particularly a novel thermochemical storage. In this approach, the OGS sets energy storage targets, but rather than merely guiding the MPC, it imposes constraints that the MPC must respect. This distinction results in a different system dynamics where the MPC operates within stricter boundaries, leading to potentially longer computation times and infeasibilities.
\\
\cite{saletti_smart_2022} proposed an approach that combines the long-term constraints of a multi-vector system with the short-term operational constraints. Their work introduces a three-module control system. The Distribution module oversees the final energy delivery to the end-user and provides energy demand forecasts using a dynamic programming approach. The Short-term supervisory controls the energy system's actions using a MILP model that takes into account the system state, forecasts, and long-term constraints. The time horizon is sufficiently short to obtain a quick solution. Finally, the Long-term supervisory establishes the long-term constraints through a linear programming problem on a daily basis. This hierarchical approach has also been recently used for multiple power-to-gas systems sharing a long term storage and connected to a district heating network \citep{MARZI2024100143}.
\\

Regarding the continuous approaches, \cite{abomazid_optimal_2022} crafted an EMS specifically tailored to tackle challenges arising from non-constant electrolyzer efficiency and seasonal fluctuations in renewable energy production. Through the implementation of a rolling-horizon methodology, their innovative EMS adeptly managed dynamic variables, underscoring its adaptability and efficacy in diverse operational contexts. Notably, their approach introduced a Z-score factor to signal seasonal trends in electricity prices to the EMS, marking an improvement in the integration of long-term strategy within a short-term vision.
\\
\cite{darivianakis_data-driven_2017} and \cite{thaler_hybrid_2023} studied the integration of a stored energy value into a traditional EMS. Specifically, \cite{thaler_hybrid_2023} developed a hybrid MPC strategy for a renewable microgrid with seasonal hydrogen storage. This EMS assigns a cost value to the hydrogen stored, serving as an incentive for long-term energy shifts. The study highlighted that even a simple constant value for stored hydrogen significantly enhances EMS performance. Their method determines the optimal hydrogen storage value by evaluating various scenarios, emphasizing the balance between production and consumption throughout the year.
\cite{darivianakis_data-driven_2017} introduced a more sophisticated approach, utilizing historical data to dynamically adjust the value of energy stored over time. This method involves constructing bounds around the optimal charging trajectory and developing a piece-wise affine approximation of the storage value function at each time step. The resulting multistage stochastic optimization problem aims to minimize the total energy consumption while ensuring that the system operates close to global optimality. This approach is particularly relevant for systems with significant seasonal variations.
\\
Regarding EMS strategies including long term management but not in the field of hydrogen production, \cite{cuisinier_new_2022,cuisinier_impact_2023} presented a rolling horizon approach based on a MILP model that combines the short term, evaluated hourly, and the long term, evaluated at an arbitrarily longer interval.
The sole decision variable for the long term becomes the variation in the State of Energy (SOE) of the long-term storage, a thermal pit storage associated to district heating in their case. This variable is employed by Cost Functions (CF), which represent the operational cost of the system for a given time interval and for a variation of storage SOE. These CF are precalculated based on numerous MILP models over Representative Periods (RP) of the long term horizon.
\\

Based on the analysis of the state-of-the-art, none of the papers examined cope with industrial constraints such as maximal CO2 content of the produced H2 or minimal H2 rate of production. The most recent and advanced approaches for EMS tailored specifically to hydrogen production in industrial settings are based on hierarchical approaches, that may have difficulties to cope with such constraints. In the field of district heating, the methodological continuous approach of \cite{cuisinier_new_2022} is promising in order to both cope with the long term management of the salt cavern and these industrial constraints.
